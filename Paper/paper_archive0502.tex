\documentclass[12pt,letterpaper]{report}
\usepackage[utf8]{inputenc}
\usepackage[english]{babel}
\usepackage{amsmath}
\usepackage{amsfonts}
\usepackage{amssymb}
\usepackage{graphicx}
\usepackage{upgreek}
\usepackage{apacite}
\usepackage{ntheorem}
\usepackage{enumitem}
\usepackage{gensymb}
\graphicspath{ {./Figures/} }
\newtheorem{hyp}{Hypothesis}
\usepackage[left=2cm,right=2cm,top=2cm,bottom=2cm]{geometry}
\author{
	Kho, Lee\\
	\textit{ltk224}
	\and
	Peeters, Jonas\\
	\textit{jp5642}
}
\title{DS-GA 1013: Project\\Intelligible Principal Component Analysis}

\usepackage[strings]{underscore}
\usepackage{fancyhdr}
\fancypagestyle{plain}{
    \fancyhf{}
    \lhead{Kho \& Peeters}
    \chead{\thepage}
    \rhead{DS-GS 1013}
}
\pagestyle{plain}

\setlength\parindent{0pt}

%% Probability operators and functions
%
% \def \P{\mathrm{P}}
\def \P{\mathrm{P}}
\def \E{\mathrm{E}}
\def \Var{\mathrm{Var}}
\let\var\Var
\def \Cov {\mathrm{Cov}} \let\cov\Cov
\def \Corr {\mathrm{Corr}}
\def \MSE {\mathrm{MSE}} \let\mse\MSE
\def \sgn {\mathrm{sgn}}
\def \R {\mathbb{R}}
\def \C {\mathbb{C}}
\def \N {\mathbb{N}}
\def \Z {\mathbb{Z}}
\def \cV {\mathcal{V}}
\def \cS {\mathcal{S}}
\newcommand{\RR}{\ensuremath{\mathbb{R}}}
\DeclareMathOperator*{\argmin}{arg\,min}
\DeclareMathOperator*{\argmax}{arg\,max}
\newcommand{\red}[1]{\textcolor{red}{#1}}
\newcommand{\blue}[1]{\textcolor{blue}{#1}}
\newcommand{\green}[1]{\textcolor{ForestGreen}{ #1}}
\newcommand{\fuchsia}[1]{\textcolor{RoyalPurple}{ #1}}
\DeclareMathOperator{\Norm}{\mathcal{N}}

%
%% Probability distributions
%
%\def \Bern    {\mathrm{Bern}}
%\def \Binom   {\mathrm{Binom}}
%\def \Exp     {\mathrm{Exp}}
%\def \Geom    {\mathrm{Geom}}
%\def \Norm    {\mathcal{N}}
%\def \Poisson {\mathrm{Poisson}}
%\def \Unif    {\mathrm {U}}
%
\newcommand{\bdb}[1]{\textcolor{red}{#1}}

\newcommand{\ml}[1]{\mathcal{ #1 } }
\newcommand{\wh}[1]{\widehat{ #1 } }
\newcommand{\wt}[1]{\widetilde{ #1 } }
\newcommand{\conj}[1]{\overline{ #1 } }
\newcommand{\rnd}[1]{\tilde{ #1 } }
\newcommand{\rv}[1]{ \rnd{ #1}  }
\newcommand{\rx}{\rnd{ x}  }
\newcommand{\ry}{\rnd{ y}  }
\newcommand{\ra}{\rnd{ a}  }
\newcommand{\rb}{\rnd{ b}  }
\newcommand{\rz}{\rnd{ z}  }
\newcommand{\rpc}{\widetilde{ pc}  }
\newcommand{\rndvec}[1]{\vec{\rnd{#1}}}


\def \cnd {\, | \,}
\def \Id { I }
\def \J {\mathbf{1}\mathbf{1}^T}

\newcommand{\op}[1]{\operatorname{#1}}
\newcommand{\setdef}[2]{ := \keys{ #1 \; | \; #2 } }
\newcommand{\set}[2]{ \keys{ #1 \; | \; #2 } }
\newcommand{\sign}[1]{\op{sign}\left( #1 \right) }
\newcommand{\trace}[1]{\op{tr}\left( #1 \right) }
\newcommand{\tr}[1]{\op{tr}\left( #1 \right) }
\newcommand{\inv}[1]{\left( #1 \right)^{-1} }
\newcommand{\abs}[1]{\left| #1 \right|}
\newcommand{\sabs}[1]{| #1 |}
\newcommand{\keys}[1]{\left\{ #1 \right\}}
\newcommand{\sqbr}[1]{\left[ #1 \right]}
\newcommand{\sbrac}[1]{ ( #1 ) }
\newcommand{\brac}[1]{\left( #1 \right) }
\newcommand{\bbrac}[1]{\big( #1 \big) }
\newcommand{\Bbrac}[1]{\Big( #1 \Big)}
\newcommand{\BBbrac}[1]{\BIG( #1 \Big)}
\newcommand{\MAT}[1]{\begin{bmatrix} #1 \end{bmatrix}}
\newcommand{\sMAT}[1]{\left(\begin{smallmatrix} #1 \end{smallmatrix}\right)}
\newcommand{\sMATn}[1]{\begin{smallmatrix} #1 \end{smallmatrix}}
\newcommand{\PROD}[2]{\left \langle #1, #2\right \rangle}
\newcommand{\PRODs}[2]{\langle #1, #2 \rangle}
\newcommand{\der}[2]{\frac{\text{d}#2}{\text{d}#1}}
\newcommand{\pder}[2]{\frac{\partial#2}{\partial#1}}
\newcommand{\derTwo}[2]{\frac{\text{d}^2#2}{\text{d}#1^2}}
\newcommand{\ceil}[1]{\lceil #1 \rceil}
\newcommand{\Imag}[1]{\op{Im}\brac{ #1 }}
\newcommand{\Real}[1]{\op{Re}\brac{ #1 }}
\newcommand{\norm}[1]{\left|\left| #1 \right|\right| }
\newcommand{\norms}[1]{ \| #1 \|  }
\newcommand{\normProd}[1]{\left|\left| #1 \right|\right| _{\PROD{\cdot}{\cdot}} }
\newcommand{\normTwo}[1]{\left|\left| #1 \right|\right| _{2} }
\newcommand{\normTwos}[1]{ \| #1  \| _{2} }
\newcommand{\normZero}[1]{\left|\left| #1 \right|\right| _{0} }
\newcommand{\normTV}[1]{\left|\left| #1 \right|\right|  _{ \op{TV}  } }% _{\op{c} \ell_1} }
\newcommand{\normOne}[1]{\left|\left| #1 \right|\right| _{1} }
\newcommand{\normOnes}[1]{\| #1 \| _{1} }
\newcommand{\normOneTwo}[1]{\left|\left| #1 \right|\right| _{1,2} }
\newcommand{\normF}[1]{\left|\left| #1 \right|\right| _{\op{F}} }
\newcommand{\normLTwo}[1]{\left|\left| #1 \right|\right| _{\ml{L}_2} }
\newcommand{\normNuc}[1]{\left|\left| #1 \right|\right| _{\ast} }
\newcommand{\normOp}[1]{\left|\left| #1 \right|\right|  }
\newcommand{\normInf}[1]{\left|\left| #1 \right|\right| _{\infty}  }
\newcommand{\proj}[1]{\mathcal{P}_{#1} \, }
\newcommand{\diff}[1]{ \, \text{d}#1 }
\newcommand{\vc}[1]{\boldsymbol{\vec{#1}}}
\newcommand{\rc}[1]{\boldsymbol{#1}}
\newcommand{\vx}{\vec{x}}
\newcommand{\vy}{\vec{y}}
\newcommand{\vz}{\vec{z}}
\newcommand{\vu}{\vec{u}}
\newcommand{\vv}{\vec{v}}
\newcommand{\vb}{\vec{\beta}}
\newcommand{\va}{\vec{\alpha}}
\newcommand{\vaa}{\vec{a}}
\newcommand{\vbb}{\vec{b}}
\newcommand{\vg}{\vec{g}}
\newcommand{\vw}{\vec{w}}
\newcommand{\vh}{\vec{h}}
\newcommand{\vnu}{\vec{\nu}}
\newcommand{\rvnu}{\vc{\nu}}
\newcommand{\vbeta}{\vec{\beta}}
\newcommand{\valpha}{\vec{\alpha}}
\newcommand{\vgamma}{\vec{\gamma}}
\newcommand{\veta}{\vec{\eta}}
\newcommand{\rw}{\rnd{w}}
\newcommand{\rvv}{\rndvec{v}}
\newcommand{\rvw}{\rndvec{w}}
\newcommand{\rvx}{\rndvec{x}}
\newcommand{\rvy}{\rndvec{y}}
\newcommand{\rvz}{\rndvec{z}}
\newcommand{\rvX}{\rndvec{X}}

\newtheorem{theorem}{Theorem}[section]
% \declaretheorem[style=plain,qed=$\square$]{theorem}
\newtheorem{corollary}[theorem]{Corollary}
\newtheorem{definition}[theorem]{Definition}
\newtheorem{lemma}[theorem]{Lemma}
\newtheorem{remark}[theorem]{Remark}
\newtheorem{algorithm}[theorem]{Algorithm}

% \theoremstyle{definition}
%\newtheorem{example}[proof]{Example}
%\declaretheorem[style=definition,qed=$\triangle$,sibling=definition]{example}
%\declaretheorem[style=definition,qed=$\bigcirc$,sibling=definition]{application}

%
%% Typographic tweaks and miscellaneous
%\newcommand{\sfrac}[2]{\mbox{\small$\displaystyle\frac{#1}{#2}$}}
%\newcommand{\suchthat}{\kern0.1em{:}\kern0.3em}
%\newcommand{\qqquad}{\kern3em}
%\newcommand{\cond}{\,|\,}
%\def\Matlab{\textsc{Matlab}}
%\newcommand{\displayskip}[1]{\abovedisplayskip #1\belowdisplayskip #1}
%\newcommand{\term}[1]{\emph{#1}}
%\renewcommand{\implies}{\;\Rightarrow\;}

% mathcal
\newcommand{\cF}{\mathcal{F}}
\newcommand{\cN}{\mathcal{N}}
\newcommand{\cT}{\mathcal{T}}
\newcommand{\cC}{\mathcal{C}}
\newcommand{\cZ}{\mathcal{Z}}
\newcommand{\cD}{\mathcal{D}}
\newcommand{\cB}{\mathcal{B}}

% Transpose 
\def\sT{{\sf T}}
\def\deg{{\rm deg}}

% Trace and Variance
\def\Tr{{{\rm Tr}}}
\def\Var{{{\rm Var}}}
\def\Cov{{{\rm Cov}}}


\def\eps{{\varepsilon}}

% Linear algebra
\def\Span{{{\rm Span}}}
\def\dim{{{\rm dim}}}
\def\Ker{{{\rm Ker}}}
\def\Im{{{\rm Im}}}
\def\Id{{{\rm Id}}}
\def\rank{{{\rm rank}}}
\def\Sp{{{\rm Sp}}}
\def\Diag{{{\rm Diag}}}

% Mean-square errors
\def\MMSE{{{\rm MMSE}}}
\def\MSE{{{\rm MSE}}}
\def\DMSE{{{\rm DMSE}}}
\def\mmse{{{\rm mmse}}}

% Probability
\renewcommand{\P}{\mathbb{P}}
\newcommand{\iid}{\overset{\text{\tiny i.i.d.}}{\sim}}
\def\Dtv{\text{D}_{\text{\scriptsize TV}}}
\def\Dkl{\text{D}_{\text{\scriptsize KL}}}

% Usual laws
\def\Poi{\text{Poi}}
\def\Pois{\text{Pois}}
\def\Bin{\text{Bin}}
\def\Ber{\text{Ber}}
\def\Unif{\text{Unif}}

% Double brackets
\def\llbracket{[\hspace{-.10em} [ }
\def\rrbracket{ ] \hspace{-.10em}] }

\def\1{\mathds{1}}

% Math-bold
\newcommand{\bbf}[1]{\mathbf{#1}}

% hat
\newcommand{\what}[1]{\widehat{#1}}

% widebar
\DeclareFontFamily{U}{mathx}{\hyphenchar\font45}
\DeclareFontShape{U}{mathx}{m}{n}{<-> mathx10}{}
\DeclareSymbolFont{mathx}{U}{mathx}{m}{n}
\DeclareMathAccent{\widebar}{0}{mathx}{"73}

% def eq
\newcommand{\defeq}{\overset{\rm def}{=}}

% Frobenius norm
\newcommand{\frob}[1]{\langle #1 \rangle_F }

% Vectorized matrix
\def\vect{{\rm vec}}

\begin{document}
\maketitle
% \section*{Outline}
% \begin{enumerate}
% \item Introduction\\
% \textit{Jonas: I feel like this part just needs some polishing, but everything we want it to do is already there.}
% \item State of the Art\\
% \textit{Overall this part should be kept for name-dropping and connecting the different papers based on their time line, core ideas and findings.}
% \begin{enumerate}
%     \item Models
%     \item Evaluation criteria
% \end{enumerate}
% \item Methodology
% \begin{enumerate}
%     \item Dataset 
%     \item Models
%     \begin{enumerate}
%         \item Thresholding
%         \item SCoT
%         \item SCoTLASS
%         \item Vines\\
%         \textit{Jonas: This one should be kept on the down low. We should spend minimal time explaining how it works and the results it yielded.}
%     \end{enumerate}
%     \item Evaluation criteria
%     \begin{enumerate}
%         \item Cumulative Explained Variance\\
%         \textit{Tables illustrating CEV versus components}
%         \item Statistical dispersion of explained variance of each principal components\\
%         \textit{Gini coefficient for each approach}
%         \item Metric for sparsity of our principal components\\
%         \textit{\cite{Hurley2008}, Check graph on conclusion's page. $\kappa_4$ seems to be a good contender since it heavily penalizes non-zero values. Maybe too aggressive, but we have a lot of choice in this paper.}
%         \item{Algorithm efficiency}
%     \end{enumerate}
% \end{enumerate}
% \item Results
% \begin{enumerate}
%     \item Table evaluation metrics per method of several hyper parameters
%     \item Interpretation of principal component for genes\\
%     \textit{Jonas: This might be challenging. I suggest we only do this for our "best" performing SPCA method.}
% \end{enumerate}
% \item Discussion
% \begin{enumerate}
%     \item Zhou's SPCA\\
%     \textit{If we don't have time to include it in our paper we can refer to it here and the findings in Zhou's paper.}
%     \item Shortly summarize the advantages and disadvantages or Sparse PCA in general and give a range of possible applications where Sparse PCA might be useful.
%     \item Enumerate interesting further research that might follow from this paper.
% \end{enumerate}
% \end{enumerate}

\section*{Introduction}

Principal component analysis (PCA) is a popular dimension-reduction technique that has wide applications across many fields. The goal of PCA is to reduce the dimension of the data while preserving as much variance as possible. PCA does this by computing orthogonal linear combinations of the original variables, called the principal components (PCs), such that the variance along these components are maximized.  The resulting PCs are ordered from highest variance to lowest variance, and thus the first few principal components will retain the most variance in the data.\\
\\
One major limitation of PCA is that the resulting linear combinations of the original $p$ variables are difficult to interpret. This is due to the fact that, in most cases, the loadings of the resulting PCs are nonzero. This is particularly challenging when the number of variables $p$ is much larger than the number of examples $n$ (commonly called “High Dimension, Low Sample Size” data, or HDLSS), which is the case for many modern data sets. Consequently, much effort has gone into developing methods of approximating PCs with fewer than the $p$ original variables.\\
\\
In this paper, we will explore several sparse PCA methods that have been developed over the past few decades and discuss the benefits and drawbacks of each method. We will evaluate each method on a breast tissue data set compiled by JP Marques de Sá at INEB-Instituto de Engenharia Biomédica.

\section*{State of the Art}
% PCA
Jolliffe \citeyear{Jolliffe1986} introduced the ubiquitous dimension-reduction technique PCA in 1986. It has since been a staple in statistics curricula and academic research. As previously mentioned, one of the most relevant drawbacks to using PCA is the difficulty of interpreting separate principal components as they are a linear combination of all the original variables. Since its creation, several attempts have been made at making the principal components more intelligible.\\
\\
% Thresholding
One of those approaches frequently applied is setting loadings with absolute values below a certain threshold to zero. This \textit{thresholding} method could lead to misleading results for a series of reasons as described by Cadima and Jolliffe \citeyear{Cadima1995}.
% Orthogonal rotations
Another common practice based on factor  analysis is to perform and orthogonal rotation on the regular PCA components \cite{Richman1986, Richman1987}. This rotation is then optimized by maximizing simplicity criteria such as \textit{varimax} and \textit{quartimax}. Both these criteria will move towards their maximum when the loadings of the vectors are either close to $0$ or $1$. The drawback of this method is that it does not preserve the successive maximization property inherent to PCA. Moreover Jolliffe \citeyear{Jolliffe1989, Jolliffe1995} warns that orthogonal rotations often lead to non-orthogonal loadings and pairwise correlation between the principal components, and should be reserved for principal components with near-equal variance.\\
\\
% SCoT
This rotation method is two-staged in nature as it requires you to first compute PCA and then rotate the resulting PCs optimally. However, Jolliffe and Uddin \citeyear{Jolliffe2000} introduced the one-stage \textit{SCoT} approach that combines the objectives of variance maximization (like PCA) and simplification (using the varimax criteria). Contrary to a the normal orthogonal rotation approach that stays within the subspace with maximum variance defined by the regular principal directions, SCoT's use of a penalty function will make the resulting principal components divert from this maximum variance subspace and hence faces a variance-simplicity trade-off.\\
\\
% Vines
While SCoT yields increased sparsity compared to PCA, the loadings will still be nonzero values. This is an issue that both Hausman \citeyear{Hausman1982} and Vines \citeyear{Vines2000} aim to solve. Hausman introduced an integer programming solution for assembling principal components using integers loadings from a finite set (typically {$-1$,$0$,$1$}). Each element is determined sequentially as to optimize the variance explained by each component.\\ 

%The problem with this method is that it rarely results in orthogonality over the components. Vines introduces a method based on simplicity and orthogonality preserving transformation on a pair of principal components. Essentially this transformation seeks to maximize the variance contained in one of the directions at cost of the other direction. From a starting set of orthogonal simple (integer) directions a pair of directions is selected based the highest absolute covariance. After the transformation this pair is restricted from further transformations until all trivial transformations are completed. The process can then start over eventually resulting in simple components where a few contain the most explained variance. Because of the nature of the transformation, the loadings of the resulting simple PCs will be restricted to a small set of integers such as $1$, $0$ and $-1$.\\
\\
% SCoTLASS
Jolliffe, Trendafilov, and Uddin \citeyear{Jolliffe2003} introduced \textit{SCoTLASS}, which is a modified PCA technique that is based on the least absolute shrinkage and selection operator (or LASSO). Including a LASSO penalty on the loadings helps to induce sparsity, however SCoTLASS is often computationally costly due to its non-convex nature. In addition, hyperparameter tuning for SCoTLASS can be challenging, altogether making SCoTLASS an impractical method for most applications.\\

% SPCA
One of the later developments is sparse PCA (henceforth SPCA) developed by Zou, Hastie and Tibshirani \citeyear{Zou2006}. Like SCoTLASS, this method incorporates the lasso method, but in addition, they reformulate the PCA in a regression-type optimization problem. This method achieves overall sparse loadings with limited variable overlap between the different principal components. However, unlike SCoT and SCoTLASS, SPCA results in non-orthogonal PCs. In the years following its first publication, SPCA has been improved upon in a plethora of ways \cite{Shen2008, Journee2010}.

%Initially, several simple methods of producing interpretable PCs were developed in the 1990s and early 2000s.  Vines  Another more \textit{ad hoc} way of producing sparse PCs is simple thresholding, as described by Cadima & Joliffe (1995).  While both of these methods are relatively easy to implement and are often used in practice, the resulting PCs may be misleading.\\
\\

% \section*{Simple Principal Components}
% Vines \citeyear{Vines2000} introduced the concept of simple principal component by restricting the magnitude of the loadings to integers such as $-1$, $0$ and $1$. This approached is based on the work for Hausman \citeyear{Hausman1982} who introduced a integer programming solution for assembling principal components using integers from a finite set. Each element is determined sequentially as to optimize the variance explained by each component. The problem with this method is that it rarely results in orthogonality over the components.\\
% \\
% Since it is hard to simplify principal components once they are calculated, Vines describes an alternative approach to obtaining simple components using an orthogonality preserving transformation on a pair of components. For each transformation of a pair of components, the variance of the data in the direction of one component is maximized while the variance in the other direction is minimized. This transformation is applied to a combination of principal components that has a high covariance. After this transformation, we exclude this pair of components for further transformations until all sensible transformations were made with the remaining components. We will describe the algorithm in the following paragraphs.\\
% \\
% For a data matrix $X\in\mathbb{R}^{N\times D}$e need to initialize $N$ principal components of length $D$ that are orthogonal to one another and consist out of integers. While does not propose initial components, we opt for a matrix $U$ constructed from [CONTINUE]. First, we start by calculating the variances and covariances of the data in the directions of all of our components. Secondly, we transform the two principal components with the highest covariance update them. These components are excluded from consideration until all reasonable transformations with other components are completed.\\
% \\
% Two directions are transformed using the "simplicity preserving transformation" introduced by Vines \citeyear{Vines2000}. Essentially the transformation orthogonally rotates and rescales two existing components $d_1$ and $d_2$. In general we can write this orthogonal rotation as $(f_1,f_2) = P(d_1,d_2)$ 
% \begin{equation}
% P=\begin{bmatrix}
% 1 & \|d_2\|_2^2 \beta\\\beta & -\|d_1\|_2^2
% \end{bmatrix}
% \end{equation}
% Similarly the covariance matrix $V^*$ of $f_1$ and $f_2$ can be written as the transformed covariance matrix $V$ of  $d_1$ and $d_2$:\footnote{Where $\Sigma$ is the covariance matrix of the data.}
% \begin{equation}
% V^* &= P^TVP\quad\text{with}\quad V = \begin{bmatrix}-& d_1 &-\\-&d_2&-\end{bmatrix}\Sigma\begin{bmatrix}\mid & \mid\\d_1^T & d_2^T\\\mid & \mid\end{bmatrix}
% \end{equation}
% When want this transformation $P(\beta)$ to yield a $V^*$ where the variance explained by the first direction $f_1$ is maximized at the cost of the variance explained by the second direction $f_2$ using the parameter $\beta$. Because we want to have integer vector before and after transformation we impose restrictions on $\beta^*$:
% \begin{equation}
%     \beta^*\in\bigg\{i/2^k, 2^k/i: i\in\{-2^k,-2^k+1,\ldots,2^k\}\text{ and }k\in\mathbb{Z}\bigg\}
% \end{equation}
% For example, if $k=1$, then $\beta^*\in\{-1, -1/2, 0, 1/2, 1\}$. Using this $\beta^*$ we retrieve the following new directions:
% \begin{equation}
%     \text{For }|\beta^*|\leq 1:\ \begin{cases}
%     f_1 = 2^kd_1 + 2^k\beta^*d_2\\
%     f_2 = 2^k\beta^*\|d_2\|^2-2^k\|d_1\|^2d_2
%     \end{cases}
% \end{equation}
% \begin{equation}
%     \text{For }|\beta^*|> 1:\ \begin{cases}
%     f_1 = 2^kd_1/\beta^* + 2^kd_2\\
%     f_2 = 2^k\|d_2\|^2-2^k\|d_1\|^2d_2/\beta^*
%     \end{cases}
% \end{equation}

\section*{Methodology}

%How did you tackle the question? How did you modify existing methods? What datasets did you use? What theoretical tools did you apply? If you deviated from your original plan, explain why.

\subsection*{Selected Sparse PCA Methods}

We implemented the following three methods to compare results across different sparse PCA approaches.\\
\begin{enumerate}[nolistsep]\\
    \item Simple Component Technique (SCoT)
    \item Simple Component Technique-LASSO (SCoTLASS)
    \item Sparse Principal Component Analysis (SPCA)\\
\end{enumerate}

Let $X \in \mathbb{R}^{n\times p}$ be a data set with $n$ samples and $p$ features and covariance matrix $\Sigma \in \mathbb{R}^{p\times p}$. SCoT seeks to solve the following constrained maximization problem for some regularization parameter $\lambda$ (here, $S(\mathbf{v})$ is the varimax rotation of a vector $\mathbf{v}$):

\begin{equation}
\begin{aligned}
\arg\max_{v_{i}} \quad & \mathbf{v_{i}^{T}}\Sigma \mathbf{v_{i}} + \lambda S(\mathbf{v_{i}})\\
\textrm{s.t.} \quad & \lVert \mathbf{v_{i}} \rVert_{2} = 1 \text{ and } \mathbf{v_{i}^{T}}\mathbf{v_{j}} = 0 \text{ for } j < i < p \\
\textrm{where} \quad & S(\mathbf{v_{i}}) = \frac{1}{p}\left[\sum_{k=1}^{p} \mathbf{v_{k}}^4- (\sum_{k=1}^{p} \mathbf{v_{k}}^2)^2 \right]\\ 
\end{aligned}
\end{equation}

The effect of including the varimax criterion in the SCoT objective function is that the solution seeks to maximize the variance of the squared loadings of the resulting components, driving loadings towards zero or $\pm{1}$. We optimized this SCoT objective using projected gradient ascent.\\

The SCoTLASS method solves the following constrained maximization problem for some regularization parameter $c$:

\begin{equation}
\begin{aligned}
\arg\max_{v_{i}} \quad & \mathbf{v_{i}^{T}}\Sigma \mathbf{v_{i}}\\
\textrm{s.t.} \quad & \lVert \mathbf{v_{i}} \rVert_{2} = 1 \text{, }
            \lVert \mathbf{v_{i}} \rVert_{1} < c \text{ and } \mathbf{v_{i}^{T}}\mathbf{v_{j}} = 0 \text{ for }  j < i < p \\
\end{aligned}
\end{equation}

Similarly to SCoT, we used projected gradient ascent to optimize SCoTLASS. However, the SCoTLASS problem involves an L1 penalty constraint, which is not differentiable at every point and thus poses an issue when using projected gradient ascent. \\

In order to remedy this, we approximated $\lVert \mathbf{v}\rVert_{1}$ as $\mathbf{v^T}\tanh{(\gamma \mathbf{v})}$ for a large $\gamma$ (specifically, $\gamma = 1000$) and reformatted the objective function to include a smooth exterior penalty function $P(x) = 0.5x(1+\tanh{(\gamma x)})$. This procedure is described in detail by Trendafilov \citeyear{Trendafilov2004}. These modifications result in the following adapted SCoTLASS optimization problem for some regularization parameter $t$ and large $\mu$ (we used $\mu = 1000$):

\begin{equation}
\begin{aligned}
\arg\max_{v_{i}} \quad & \mathbf{v_{i}^{T}}\Sigma \mathbf{v_{i}} - \mu P(\mathbf{v_{i}^T} \tanh{(\gamma\mathbf{v_{i}})} - t)\\
\textrm{s.t.} \quad & \lVert \mathbf{v_{i}} \rVert_{2} = 1 \text { and } \mathbf{v_{i}^{T}}\mathbf{v_{j}} = 0 \text{ for }  j < i < p \\
\end{aligned}
\end{equation}

SPCA solves the following constrained maximization problem.

\begin{equation}
\begin{aligned}
\arg\max_{v_{i}} \quad & \text{FILL IN}\\
\textrm{s.t.} \quad & \text{FILL IN}\\
\end{aligned}
\end{equation}


\subsection*{Data Set}

Although we originally planned on evaluating each sparse PCA method on a gene expression data set with over 20,000 variables, we decided to use a smaller data set to make it easier to interpret results. In addition, certain methods (such as SCoTLASS) are computationally intractable for high-dimension data sets such as gene expression data sets. Ultimately, we tested each method using a data set of measurements taken on recently excised breast tissue. The data set had 9 variables and 106 examples (see Table \ref{tab1:dta_vars} for descriptions of each variable). \\

\subsection*{Evaluation Criteria}

We evaluated these methods based on three criteria: sparsity, percentage of explained variance (PEV), and efficiency and scalability.\\

\textit{Sparsity:} Since the sparse PCA methods we evaluated often result in nonzero loadings that are very close to zero, we cannot simply count the number of zero loadings as a measure of sparsity. For this reason, we used the \textit{gini coefficient} to measure sparsity of loadings. The gini coefficient is a simple measure of weight distribution across a vector. The coefficient takes a value between 0 and 1 inclusive, with a 0 value indicating total equality of loadings and a value of 1 indicating total inequality (i.e. there is only one nonzero loading). Thus, a larger gini coefficient corresponds with higher sparsity. \\

\textit{Percentage of explained variance (PEV):} An effective sparse PCA method will achieve sparsity while maintaining as much explained variance in the first few principal components as possible. For each sparse PCA method, we calculated cumulative PEV (CPEV) of the PCs and plotted the results for comparison purposes. \\

\textit{Efficiency and scalability:} Certain sparse PCA methods work well with small data sets with few variables, but become computationally impractical with higher-dimensional data sets. For each method, we measured run time on the breast tissue data set and analyzed level of complexity to assess the suitability of each method for different data set types.\\

%Results: What results did you obtain? Do they make sense? Provide a thorough analysis. Negative results are completely fine (they can be very valuable!).

\section*{Results}

The table below gives a summary of the results of the different sparse PCA methods for varying regularization parameters.\\

[TABLE HERE]\\


We find that, as we begin to constrain loadings, both SCoT and SCoTLASS shift the majority of weight from variable 4 to variable 7. Variable 6 also surprisingly jumps from 6th largest weight in PCA to 3rd largest weight. This indicates that there may be more predictive significance in these two variables than regular PCs would suggest. \\

In terms of efficiency and runtime, the SPCA function takes considerably less time to run than SCoT and SCoTLASS. This is largely due to our use of projected gradient ascent in maximizing SCoT and SCoTLASS, which required very small step sizes over many iterations in order to for the algorithms to converge. SCoTLASS is particularly costly to optimize due to the objective function's nonconvexity. This results in possible local maxima, therefore requiring several random starting points to obtain the global maxima. \\

In optimizing SCoT and SCoTLASS, we found is quite challenging to select the best regularization parameter to yield optimal levels of sparsity and explained variance. There is little guidance from relevant literature on selecting regularization parameters for these methods.\\

In addition, as we varied the regularization parameter for SCoT, we also find that there is a sudden increase in loading sparsity, which is accompanied by a sharp decrease in explained variance. At $\lambda$ = 600, SCoT produces a PC 1 with one zero loading and most other loadings between 0.30 and 0.44 and explains approximately 60\% of variance in the data. At $\lambda$ = 650 however, the PC 1 that results has six zero loadings, with one of the three nonzero loadings carrying nearly all the weight with a 0.98 coefficient, and explains only 20\% of variance in the data. This phenomenon makes SCoT an even more challenging model to tune. 

% Pros of SCoT:
% - PCs are orthogonal
% - Computationally simpler than SCoTLASS
% Cons of SCoT:
% - difficult to tune 
%
% Pros of SCoTLASS:
% - PCs are orthogonal
% Cons of SCoTLASS:
% - Costly, requires random starts
% - difficult to tune
% - resulting PCs not as sparse as those resulting from other sparse PCA methods
% - selects at most n zero loadings (problematic when p >> n)
% Pros of SPCA:
%
% Cons of SPCA:
% - produces non orthogonal pcs

\section*{Discussion}

%Discussion: What did you find out? Do the results suggest any other interesting questions to explore?

% - explore ways to create uncorrelated sparse PCAs
% - explore other optimization methods for SCoT and SCoTLASS that are more efficient



% References
\pagebreak
\bibliographystyle{apacite}
\bibliography{references}

% Appendix
\pagebreak
\appendix
\section*{Appendix}
\begin{table}[htpb!]
\caption{Variable descriptions}
\label{tab1:dta_vars}
\begin{center}
 \begin{tabular}{||c l||}
 \hline
Variable & Variable Description \\ [0.5ex] 
 \hline\hline
 $x_{1}$ & I0 - impedivity (ohm) at zero frequency \\ 
 \hline
 $x_{2}$ & 	PA500 - phase angle at 500 KHz \\
 \hline
 $x_{3}$ & HFS - high-frequency slope of phase angle \\
 \hline
 $x_{4}$ & DA - impedance distance between spectral ends \\
 \hline
 $x_{5}$ & AREA - area under spectrum \\
 \hline
 $x_{6}$ & 	A/DA - area normalized by DA \\
 \hline
 $x_{7}$ & MAX IP - maximum of the spectrum \\
 \hline
 $x_{8}$ & 	DR - distance between I0 and real part of the maximum frequency point \\
 \hline
 $x_{9}$ & P - length of the spectral curve \\ [1ex] 
 \hline
\end{tabular}
\end{center}
\end{table}
\end{document}