\documentclass[11pt]{beamer}
\usetheme{Warsaw}
\usepackage[utf8]{inputenc}
\usepackage[english]{babel}
\usepackage{amsmath}
\usepackage{amsfonts}
\usepackage{amssymb}
\usepackage{graphicx}
\author{
	Kho, Lee
	\and
	Peeters, Jonas
}
\title{Sparse Principal Components Analysis}
% \setbeamercovered{transparent} 
% \setbeamertemplate{navigation symbols}{} 
% \logo{} 
\institute{\includegraphics{../Figures/nyu_long_black.png}} 
\date{\today} 
\subject{DS-GA 1013} 

% \begin{document}

% % \begin{frame}
% % \titlepage
% \end{frame}

% \begin{frame}\frametitle{Introduction}
% \begin{itemize}
%     \item PCA is a popular dimension reduction tool with applications across many domains; however, one drawback of the technique is that PCs are usually a linear combination of \textit{all} variables and are thus difficult to interpret
%     \begin{itemize}
%         \item This is especially challenging in high-dimension, low sample size (HDLSS) data sets (i.e. "fat" matrices)
%         \item PCA may also lead to inconsistent results in HDLSS data
%     \end{itemize}
%     \item \textbf{Sparse PCA} methods have been developed in an effort to compute more interpretable PCs with few nonzero loadings
% \end{itemize}
% \end{frame}

% \begin{frame}\frametitle{State of the Art}
% \begin{itemize}
%     \item Simple thresholding, rotation and integer programming methods are sometimes used in practice to obtain sparse PCs, though these methods have been shown to produce misleading results
% \end{itemize}
% \end{frame}

% \begin{frame}\frametitle{Methodology}
% \begin{itemize}
% \item 
% \end{itemize}
% \end{frame}

% \begin{frame}\frametitle{Results}
% \begin{itemize}
% \item 
% \end{itemize}
% \end{frame}

% \begin{frame}\frametitle{Discussion}
% \begin{itemize}
% \item 
% \end{itemize}
% \end{frame}

\end{document}
